%\title{Article HW Template}

%\documentclass[12pt]{article}
\documentclass[12pt]{scrartcl}
\nonstopmode
%\usepackage[utf-8]{inputenc}
\usepackage{graphicx} % Required for including pictures
\graphicspath{{Images/}} % Specifies the directory where pictures are stored
\usepackage[figurename=Figure]{caption}
\usepackage{float}    % For tables and other floats
\usepackage{verbatim} % For comments and other
\usepackage{amsmath}  % For math
\usepackage{amssymb}  % For more math
\usepackage{fullpage} % Set margins and place page numbers at bottom center
\usepackage{paralist} % paragraph spacing
\usepackage{listings} % For source code
\usepackage{subfig}   % For subfigures
%\usepackage{physics}  % for simplified dv, and 
\usepackage{enumitem} % useful for itemization
\usepackage{siunitx}  % standardization of si units

\usepackage{tikz} % Useful for drawing plots
\usepackage{pgfplotstable}
\usepackage{pgfplots}
\pgfplotsset{compat=1.5}

\definecolor{mygreen}{RGB}{28,172,0} % color values Red, Green, Blue
\definecolor{mylilas}{RGB}{170,55,241}
\lstset{language=Matlab,%
	%basicstyle=\color{red},
	breaklines=true,%
	morekeywords={matlab2tikz},
	keywordstyle=\color{blue},%
	morekeywords=[2]{1}, keywordstyle=[2]{\color{black}},
	identifierstyle=\color{black},%
	stringstyle=\color{mylilas},
	commentstyle=\color{mygreen},%
	showstringspaces=false,%without this there will be a symbol in the places where there is a space
	%numbers=left,%
	%numberstyle={\tiny \color{black}},% size of the numbers
	%numbersep=9pt, % this defines how far the numbers are from the text
	emph=[1]{for,end,break},emphstyle=[1]\color{red}, %some words to emphasise
	%emph=[2]{word1,word2}, emphstyle=[2]{style},    
}

\title{ELEC 340 Assignment 1}

\usepackage{circuitikz}
\usepackage{etoolbox}
\makeatletter

\pgfcircdeclarebipole{}
{\ctikzvalof{bipoles/tline/height}}{tline}
{\ctikzvalof{bipoles/tline/height}}
{\ctikzvalof{bipoles/tline/width}}
{   
	%% First find distance from startpoint to endpoint
	\pgfpointdiff{\pgfpointanchor{\ctikzvalof{bipole/name}start}{center}}
	{\pgfpointanchor{\ctikzvalof{bipole/name}end}{center}}
	\pgfmathparse{veclen(\the\pgf@x,\the\pgf@y)}
	%% The coordinate system has been changed so that the origin is at the midpoint and
	%% the line is along the x axis. So shift back by half the length of the line, and 
	%% make the cylinder of width roughly the length of the line, with a 40pt setback
	%% on each side.
	\pgftransformxshift{\pgfmathresult/2-30pt}
	\pgf@circ@res@left=\dimexpr-\pgfmathresult pt+40pt\relax
	%% Here is the original function, copied directly from the source of circuittikz, 
	%% down to next %%
	\pgf@circ@res@step=.2\pgf@circ@res@right % half x axis
	\pgfsetlinewidth{\pgfkeysvalueof{/tikz/circuitikz/bipoles/thickness}\pgfstartlinewidth}
	\pgfpathellipse{\pgfpoint{\pgf@circ@res@right-\pgf@circ@res@step}{0}}
	{\pgfpoint{\pgf@circ@res@step}{0}}
	{\pgfpoint{0}{-\pgf@circ@res@up}}
	\pgfpathmoveto{\pgfpoint{\pgf@circ@res@right-\pgf@circ@res@step}{\pgf@circ@res@up}}
	\pgfpathlineto{\pgfpoint{\pgf@circ@res@left+\pgf@circ@res@step}{\pgf@circ@res@up}}
	\pgfpatharc{-90}{90}{-\pgf@circ@res@step and -\pgf@circ@res@up}
	\pgfpathlineto{\pgfpoint{\pgf@circ@res@right-\pgf@circ@res@step}{\pgf@circ@res@down}}
	%% I have to fill the figure to block out the original line
	\pgfsetfillcolor{white}
	\pgfusepath{draw,fill}
	%% Redraw part of the line that gets blocked by the cylinder by mistake
	\pgfpathmoveto{\pgfpoint{\pgf@circ@res@right-2*\pgf@circ@res@step}{0pt}}
	\pgfpathlineto{\pgfpoint{\pgf@circ@res@right}{0pt}}
	\pgfusepath{draw}
}
\begin{document}


\begin{center}
	\hrule
	\vspace{.4cm}
	{\textbf { \large ELEC 340 --- Applied Electromagnetics and Photonics}}
\end{center}
{\textbf{Name:}\ David Li \hspace{\fill} \textbf{Student Number:} \ V00818631  \\
{\textbf{Due Date:} Thursday, 11 January 2018, 11:30 AM \hspace{\fill} \textbf{Assignment 1} \today \\
\hrule

\paragraph{Problem 1}
The voltage of an electromagnetic wave traveling on a transmission line is given by, 
\[
v(z,t) = 5e^{\alpha z} \sin(4\pi \times 109t - 20 \pi z) \ \si{\volt}
\]
where z is the distance in meters from the generator. 

\begin{figure}[H]
	\centering
	\tikzstyle{block} = [draw, fill=white, rectangle, 
	minimum height=3em, minimum width=3em]
	\begin{circuitikz}
		\draw (0,4) to [twoport,t=$Z_g$](4,4);
		\draw (4,0) to[TL] (10,0);
		\draw (4,4) to[TL]  (10,4);
		\draw (10,4) to [twoport,t=$Z_L$](10,0);
		\draw (0,0) to [sI] (0,4);
		\draw (4,0) to (0,0);
	\end{circuitikz}
	\caption{Transmission line}
\end{figure}
\begin{enumerate}[label=(\alph*)]
	\item Find the frequency, wavelength, and phase velocity of the wave.
	\begin{align*}
	& \omega = 4 \pi \times 109 \si{\radian / \si{\second}} \rightarrow \text{frequency} = \frac{\omega}{2\pi} = 2 \times 109 = 218 \si{\hertz} \\
	& \beta=\frac{2\pi}{\lambda}= 20\pi \rightarrow \text{wavelength} =\lambda = {2\pi}{\beta} = \frac{2 \pi} {20 \pi} = 0.1 \si{\meter} \\
	& \text{phase velocity} =  f \lambda = \frac{\omega}{\beta} = \frac{4 \pi \times 109}{20 \pi} = 21.8 \si{\meter} / \si{\second}
	\end{align*}
	\item  At 𝑧 = $2 \si{\meter}$, the amplitude of the wave was measured to be $2 \si{\volt}$. Find $\alpha$. 
	When $z = 2$, $v(2,t)=2 \si{\volt} = 5e^{\alpha z} \si{\volt}$ 
\end{enumerate}

\paragraph{Problem 2} 
A laser beam travelling through fog was observed to have an intensity of 1 ($\mu \si{\watt}/ \si{\meter}^2$) at a distance of $2 \si{\meter}$ from the laser gun and an intensity of 0.2 ($\mu \si{\watt}/ \si{\meter}^2$) at a distance of $3 \si{\meter}$. Given that the intensity of an electromagnetic wave is proportional to the square of its electric-field amplitude, find the attenuation constant $\alpha$ of fog. 

Since intensity of an electromagnetic wave is s proportional to the square of its electric-field amplitude, it is:
\begin{align*}
& E(x,t) = E_0e^{-\alpha x}\cos(\omega t - \beta x) \\
& I(x,t) = [E(x,t)]^2 = [ E_0e^{-\alpha x}\cos(\omega t - \beta x) ]^2 	\\
& I(x,t) = E_0^2e^{-2\alpha x}\cos^2(\omega t - \beta x)
\end{align*}

When $x=2 \si{\meter}$, $I(x,t)=1 \mu \si{\watt}/ \si{\meter}^2$, and when $x=3$, $I(x,t)=0.2 (\mu \si{\watt}/ \si{\meter}^2)$

\begin{align*}
& \cfrac{E_0^2e^{-2\alpha(2)}}{E_0^2e^{-2\alpha(3)}} \\
&=\cfrac{1 \times 10^{-6}}{0.2 \times 10^{-6}}=5 \\
& e^{-4\alpha} \times e^{6\alpha}=e^{2\alpha}=5 \\
& \alpha = 0.8 (NP/m)
\end{align*}
\paragraph{Problem 3}
\begin{enumerate}[label=(\alph*)]
	\item  $v(t)= 9 \cos(\omega t - \pi / 3) \si{\volt}$ \newline
	$\tilde{V}=9e^{-j \pi / 3}  \si{\volt}$
	\item  $v(t)= 12 \sin(\omega t + \pi / 4) \si{\volt}$  \newline
	Using $\cos(\omega t)=\sin(\omega t + \pi/2)$,  $\sin(\omega t + \pi / 4 + \pi /2 - \pi/2)= \cos(\omega t - \pi /4) $  \newline
	$\tilde{V}=12e^{-j \pi / 4}  \si{\volt}$
	\item  $i(x,t)= 5e^{-3x} \sin(\omega t + \pi / 6) \si{\ampere}$  \newline
	$\sin(\omega t + \pi / 6 + \pi /2 - \pi/2)= \cos(\omega t - \pi / 3) $  \newline
	$\tilde{I}=5e^{-3x}e^{-j \pi / 3} \si{\ampere}$
	\item  $ i(t)= -2\cos(\omega t + 3\pi /4) \si{\ampere} $ \newline
	$\tilde{I}=-2e^{j 3\pi / 4} =2e^{-j\pi}e^{j 3\pi / 4} =2e^{-j \pi / 4} \si{\ampere}$
	\item  $ i(t)= 4 \sin(\omega t + \pi / 3) + 3\cos(\omega t - \pi /6) \si{\ampere}$ \newline
	\begin{align*}
	i(t) & = 4 \sin(\omega t + \pi /3) + 3\cos(\omega t - \pi /6) \\
	     & = 4 \cos(\pi /2 - (\omega t + \pi /3)) + \cos(\omega t - \pi /6) \\
	     & = 4 \cos(\omega t - \pi /6 ) + 3\cos(\omega t - \pi /6) \\
	     & = 7 \cos(\omega t - \pi /6) \si{\ampere}
	\end{align*}
\end{enumerate}
\paragraph{Problem 4}
\begin{enumerate}[label=(\alph*)]
	\item $\tilde{V}=-5e^{j\pi/3} \si{\volt}$
	\begin{align*}
	& \tilde{V}=5e^{j\pi/3}e^{-j\pi} = 5e^{-2\pi/3} \\
	& v(t) = 5\cos(\omega t  - 2\pi /3)  \si{\volt}
	\end{align*}
	\item $\tilde{V}=-j6e^{-j\pi/4} \si{\volt}$
	\begin{align*}
	& \tilde{V}=-j6e^{-j\pi/4}=6e^{j3\pi/2}e^{-j\pi/4} \si{\volt} \\
	& \tilde{V}= 6e^{j5\pi /4}  \si{\volt} \\
	& v(t) = 6 \cos(\omega t + 5\pi /4) \si{\volt}
	\end{align*}
	\item $\tilde{I}=(6+j8)\si{\ampere}$
	\begin{align*}
	& \tilde{I}=(6+j8)=10e^{j53.1301^{o}} \\
	& i(t) = 10\cos(\omega t + 53.1301^o)
	\end{align*}
	\item $\tilde{I}=(-3+j2)\si{\ampere}$
	\begin{align*}
	& \tilde{I}=(-3+j2)=\sqrt{13}e^{j141.31^{o}} \\
	& i(t) = \sqrt{13}\cos(\omega t + 141.31^{o})
	\end{align*}
	\item $\tilde{I}=j\si{\ampere}$ \newline
	$i(t)=\cos(\omega t + \pi /2)$
\end{enumerate}
%\subsection*{7.1}
%Matrix M=$\xi \xi^T$ has rank one since each column of M is a multiple of non-zero vector, $\xi$
%Matrix M=$\xi \xi^T$ is symmetric because $M^T=\left(\xi \xi^T\right)^{T}=\xi \xi^T= M$ \newline
%M is positive semidefinite since for any arbitrary $x$, $x^T M x = x^T \left(\xi \xi^T \right)x=(x^T \xi)^2 \geq 0$
%\subsection*{7.7 a}
%\begin{align*}
%& f(x) = 100(x_2-x_1^2)^2 + (1-x_1)^2 \\
%& g(x) = \begin{bmatrix}
%{2} {x_{1}} - {400} {x_{1}} ({x_{2}} - {x_{1}}^{2}) - {2} & {200} {x_{2}} - {200} {x_{1}}^{2}
%\end{bmatrix}
%\end{align*}
%In all cases, using the three given data points causes the algorithm to converge to the minimizer at [1 1]. The average CPU results for $\epsilon = 10^{-6}$ is calculated for the three trial points counted over 1000 runs.
%\begin{figure}[h]
%{\renewcommand{\arraystretch}{1} %<- modify value to suit your needs
%\begin{tabular}{ccccc}
%	Initial Point & $x^*$ & Number of Iterations & Average CPU time & $f(x^*)$  \\ 
%	\hline 
%	 &  &  &  &  \\ 
%	$\begin{bmatrix} 2 	\\ 2 \end{bmatrix}$ & $\begin{bmatrix} 1.0000 	\\ 1.0000 \end{bmatrix}$  
%		& $k=24$ & 0.0301 & $1.6599 \times 10^{-19}$  \\ 
%	&  &  &  & \\
%	$\begin{bmatrix} -2 	\\ 2 \end{bmatrix}$ & $\begin{bmatrix} 1.0000 	\\ 1.0000 \end{bmatrix}$ & $k=33$  & 0.0504 & $6.0705 \times 10^{-29}$ \\ 
%	&  &  &  & \\
%	$\begin{bmatrix} 2 	\\ -2 \end{bmatrix}$&  $\begin{bmatrix} 1.0000 	\\ 1.0000 \end{bmatrix}$ & $k=23$  &  0.0333 & $1.5440 \times 10^{-20}$ \\ 
%	\hline 
%\end{tabular} 
%}
%\captionof{table}{DFP Algorithm Results}
%\end{figure}
%\subsection*{7.12}
%\begin{align*}
%& f(x) = 100(x_2-x_1^2)^2 + (1-x_1)^2 \\
%& g(x) = \begin{bmatrix}
%{2} {x_{1}} - {400} {x_{1}} ({x_{2}} - {x_{1}}^{2}) - {2} & {200} {x_{2}} - {200} {x_{1}}^{2}
%\end{bmatrix}
%\end{align*}
%
%\begin{figure}
%	{\renewcommand{\arraystretch}{1} %<- modify value to suit your needs
%		\begin{tabular}{ccccc}
%			Initial Point & $x^*$ & Number of Iterations & Average CPU time & $f(x^*)$  \\ 
%			\hline 
%			&  &  &  &  \\ 
%			$\begin{bmatrix} 2 	\\ 2 \end{bmatrix}$ & $\begin{bmatrix} 1.0000 	\\ 1.0000 \end{bmatrix}$  
%			& $k=19$ & 0.0250 & $ 4.1931 \times 10^{-19}$  \\ 
%			&  &  &  & \\
%			$\begin{bmatrix} -2 	\\ 2 \end{bmatrix}$ & $\begin{bmatrix} 1.0000 	\\ 1.0000 \end{bmatrix}$ & $k=28$  & 0.0416 & $1.3647 \times 10^{-18}$ \\ 
%			&  &  &  & \\
%			$\begin{bmatrix} 2 	\\ -2 \end{bmatrix}$&  $\begin{bmatrix} 1.0000 	\\ 1.0000 \end{bmatrix}$ & $k=20$  & 0.0279 & $1.4996 \times 10^{-18}$ \\ 
%			\hline 
%		\end{tabular} 
%	}
%	\captionof{table}{BFGS Algorithm Results}
%\end{figure}
%
%%\begin{figure}[h]
%%	{\renewcommand{\arraystretch}{1} %<- modify value to suit your needs
%%		\begin{tabular}{ccccc}
%%			Initial Point & $x^*$ & Number of Iterations & Average CPU time & $f(x^*)$  \\ 
%%			\hline 
%%			&  &  &  &  \\ 
%%			$\begin{bmatrix} 2 	\\ 2 \end{bmatrix}$ & $\begin{bmatrix} 1.0000 	\\ 1.0000 \end{bmatrix}$  
%%			& $k=24$ & 0.0468 & $1.6599 \times 10^{-19}$  \\ 
%%			&  &  &  & \\
%%			$\begin{bmatrix} -2 	\\ 2 \end{bmatrix}$ & $\begin{bmatrix} 1.0000 	\\ 1.0000 \end{bmatrix}$ & $k=28$  & 0.0604 & $1.9283 \times 10^{-17}$ \\ 
%%			&  &  &  & \\
%%			$\begin{bmatrix} 2 	\\ -2 \end{bmatrix}$&  $\begin{bmatrix} 1.0000 	\\ 1.0000 \end{bmatrix}$ & $k=23$  &  0.0483 & $1.5440 \times 10^{-20}$ \\ 
%%			\hline 
%%		\end{tabular} 
%%	}
%%	\captionof{table}{DFP Algorithm Results}
%%\end{figure}
%%\subsection*{7.12}
%%\begin{align*}
%%& f(x) = 100(x_2-x_1^2)^2 + (1-x_1)^2 \\
%%& g(x) = \begin{bmatrix}
%%{2} {x_{1}} - {400} {x_{1}} ({x_{2}} - {x_{1}}^{2}) - {2} & {200} {x_{2}} - {200} {x_{1}}^{2}
%%\end{bmatrix}
%%\end{align*}
%%
%%\begin{figure}
%%	{\renewcommand{\arraystretch}{1} %<- modify value to suit your needs
%%		\begin{tabular}{ccccc}
%%			Initial Point & $x^*$ & Number of Iterations & Average CPU time & $f(x^*)$  \\ 
%%			\hline 
%%			&  &  &  &  \\ 
%%			$\begin{bmatrix} 2 	\\ 2 \end{bmatrix}$ & $\begin{bmatrix} 1.0000 	\\ 1.0000 \end{bmatrix}$  
%%			& $k=19$ & 0.0692 & $ 4.1931 \times 10^{-19}$  \\ 
%%			&  &  &  & \\
%%			$\begin{bmatrix} -2 	\\ 2 \end{bmatrix}$ & $\begin{bmatrix} 1.0000 	\\ 1.0000 \end{bmatrix}$ & $k=28$  & 0.1206 & $1.3647 \times 10^{-18}$ \\ 
%%			&  &  &  & \\
%%			$\begin{bmatrix} 2 	\\ -2 \end{bmatrix}$&  $\begin{bmatrix} 1.0000 	\\ 1.0000 \end{bmatrix}$ & $k=20$  & 0.0879 & $1.4996 \times 10^{-18}$ \\ 
%%			\hline 
%%		\end{tabular} 
%%	}
%%	\captionof{table}{BFGS Algorithm Results}
%%\end{figure}
%Although BFGS and DFP have comparable solution accuracy, the BFGS algorithm is a little more effcient in terms of average CPU time than the DFP algorithm in the three trial cases.
\end{document}
%%%%%%%%%%%%%%%%%%%%%%%%%%%%%%

%%%%%%%%%%%%%%%%
COMMON COMMANDS:
%%%%%%%%%%%%%%%%
% IMAGES
\begin{figure}[H]
	\begin{center}
		\includegraphics[width=0.6\textwidth]{RTL_SCHEM.png}
	\end{center}
	\caption{A screenshot of the RTL Schematics produced from the Verilog code.}
	\label{RTL}
\end{figure}

% SUBFIGURES IMAGES
\begin{figure}[H]
	\centering
	\subfloat[LED4 Period]{\label{fig:Per4}\includegraphics[width=0.4\textwidth]{period_led4.png}} \\                
	\subfloat[LED5 Period]{\label{fig:Per5}\includegraphics[width=0.4\textwidth]{period_led5.png}}
	\subfloat[LED6 Period]{\label{fig:Per6}\includegraphics[width=0.4\textwidth]{period_led6.png}}
	\caption{Period of LED blink rate captured by osciliscope.}
	\label{fig:oscil}
\end{figure}

% INSERT SOURCE CODE
\lstset{language=Verilog, tabsize=3, backgroundcolor=\color{mygrey}, basicstyle=\small, commentstyle=\color{BrickRed}}
\lstinputlisting{MODULE.v}

% TEXT TABLE
\begin{table}
	\begin{center}
		\begin{tabular}{|l|c|c|l|}
			x & x & x & x \\ \hline
			x & x & x & x \\
			x & x & x & x \\ \hline
		\end{tabular}
		\caption{Caption}
		\label{label}
	\end{center}
\end{table}

% MATHMATICAL ENVIRONMENT
$ 8 = 2 \times 4 $

% CENTERED FORMULA
\[  \]

% NUMBERED EQUATION
\begin{equation}

\end{equation}

% ARRAY OF EQUATIONS (The splat supresses the numbering)
\begin{align*}

\end{align*}

% NUMBERED ARRAY OF EQUATIONS
\begin{align}

\end{align}

% ACCENTS
\dot{x} % dot
\ddot{x} % double dot
\tilde{x} % bar
\tilde{x} % tilde
\vec{x} % vector
\hat{x} % hat
\acute{x} % acute
\grave{x} % grave
\breve{x} % breve
\check{x} % dot (cowboy hat)

% FONTS
\mathrm{text} % roman
\mathsf{text} % sans serif
\mathtt{text} % Typewriter
\mathbb{text} % Blackboard bold
\mathcal{text} % Caligraphy
\mathfrak{text} % Fraktur

\textbf{text} % bold
\textit{text} % italic
\textsl{text} % slanted
\textsc{text} % small caps
\texttt{text} % typewriter
\underline{text} % underline
\emph{text} % emphasized

\begin{tiny}text\end{tiny} % Tiny
\begin{scriptsize}text\end{scriptsize} % Script Size
\begin{footnotesize}text\end{footnotesize} % Footnote Size
\begin{small}text\end{small} % Small
\begin{normalsize}text\end{normalsize} % Normal Size
\begin{large}text\end{large} % Large
\begin{Large}text\end{Large} % Larger
\begin{LARGE}text\end{LARGE} % Very Large
\begin{huge}text\end{huge}   % Huge
\begin{Huge}text\end{Huge}   % Very Huge


% GENERATE TABLE OF CONTENTS AND/OR TABLE OF FIGURES
% These seem to have some issues with the "revtex4" document class.  To use, change
% the very first line of this document to "article" like this:
% \documentclass[aps,letterpaper,10pt]{article}
\tableofcontents
\listoffigures
\listoftables

% INCLUDE A HYPERLINK OR URL
\url{http://www.derekhildreth.com}
\href{http://www.derekhildreth.com}{Derek Hildreth's Website}

% FOR MORE, REFER TO THE "LINUX CHEAT SHEET.PDF" FILE INCLUDED!

